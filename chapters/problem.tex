\newpage
\begin{center}
	\textbf{\large 2. СПЕЦИАЛЬНАЯ ЧАСТЬ}
\end{center}
\refstepcounter{chapter}
\addcontentsline{toc}{chapter}{2. СПЕЦИАЛЬНАЯ ЧАСТЬ}

\section{Характеристики оптического канала}

Рассмотрим комплекснозначный сигнал $x(t)$, который распространяется по оптическому волокну длины $L$ согласно Рисунку $4.1$.

\begin{figure}[!h]
	\begin{center}
		\includegraphics[width=300pt]{propogation.png}
		\caption{Распространение сигнала в одномодовом оптоволокне.}
		\label{nmp}
	\end{center}
\end{figure}

Нелинейное уравнение Шрёдингера описывает взаимосвязь между отправленным сигналом $x(t)=u(z=0,t)$ и полученным сигналом $y(t)=u(z=L,t)$:

\begin{equation}
	\frac{\partial u(z,t)}{\partial z} = - \frac{\alpha}{2} u(z,t) - \frac{i\beta_2}{2} \frac{\partial^2 u(z,t)}{\partial t^2} + i \gamma |u(z,t)|^2 u(z,t),
\end{equation}

где $\alpha$ -- коэффициент затухания сигнала, $\beta_2$ -- коэффициент хроматической дисперсии, $\gamma$ -- нелинейный коэффициент (ответственный за эффект Керра).

Моделирование распространения сигнала по long-haul волокну выполнено лабораторией цифровой обработки сигналов Huawei.
Моделирование выполнялось следующим образом: сигнал распространялся по закольцованному волокну длиной около $10$ километров большое количество раз ($>60$ в нашем случае), также вносились дополнительные шумы, имитирующие
Параметры моделирования выбирались следующими:

\begin{itemize}
	\item Высокая излучаемая мощность лазера: $>10$ дБм, волна $\lambda = 1550$ нм.
	\item Длина канала $L = 640$ км.
	\item Модуляция сигнала: квадратурная модуляция $16\text{QAM}$ ($1$ символ -- $4$ бита информации) и кодирование Грея (Gray mapping); диаграмма созвездия показана на Рисунке $4.2$.
	\item 1 поляризация сигнала (single-polarization)
	\item Коэффициент затухания в оптоволокне $\alpha = 0.2$ дБ/км.
	\item Линейный коэффициент хроматической дисперсии $\beta_2 = -21.683$ $\text{пс}^2$/км.
	\item Нелинейный коэффициент Керра $\gamma = 1.3$ рад / (Ват$\cdot$км)
\end{itemize}

\begin{figure}[!h]
	\begin{center}
		\includegraphics[width=300pt]{qam-gray.png}
		\caption{16QAM модуляция с Gray mapping~\cite{QAMarticle}.}
		\label{nmp}
	\end{center}
\end{figure}

\section{Модель ВОЛС}

Блок-схема модели ВОЛС (передатчик -- оптоволоконный канал -- приёмник) показана на Рисунке 4.3.

\begin{figure*}[t]
	\centering
	\includegraphics[width=475pt]{model.png}
	\caption{Блок-схема end-to-end модели системы.}
	\label{fig:block_diagram}
\end{figure*}

Передаваемый сигнал (TX signal) $x(t)$ задаётся выражением

\begin{equation}
	x(t) = \sqrt{P} \sum \limits_{k=-\infty}^{+\infty} s_k p \left( t - \frac{k}{R_s} \right),
\end{equation}

где $P$ -- мощность сигнала, $\left\{ s_k \right\}_{k \in \mathbb{Z}}$ -- последовательность комплексных чисел, а $p(t)$ -- импульсная характеристика фильтра с косинусно-квадратным спектром (RRC) и $R_s$ -- символьная скорость (в бод = $1$ бит/с).
Один кадр данных, одновременно подаваемый на обработку, представляет из себя вектор постоянной размерности $N_{sym}$: $\left( s_1, \ldots, s_{N_{sym}} \right)$ -- символы являются независимыми и имеют симметричное комплексное гауссово распределение.

Сигнал $x(t)$ предполагается переданным по оптической линии, состоящей из $N_{sp}$ участков стандартного одномодового волокна (SMF).
Каждый участок имеет длину $L_{sp}$, после каждого участка включён эрбиевый волоконный усилитель (EDFA) с коэффициентом шума $N_F$,
компенсирующий вносимое затухание сигнала.
Каждый EDFA добавляет белый гауссов шум со спектральной плотностью мощности $\left( \text{e}^{\alpha L_{sp}} - 1 \right) h \nu_s n_{sp}$~\cite{5420239}, где $h$ -- постоянная Планка, $\nu_s$ -- частота оптической несущей и $n_{sp} = 0.5 N_F (1 - \text{e}^{-\alpha L_{sp}})$ -- коэффициент спонтанного излучения.

После распространения через всю линию длины $L = N_{sp} L_{sp}$ принятый сигнал (RX signal) $y(t)$ фильтруется идеальным фильтром нижних частот (АЧХ) с полосой пропускания $2f_s$ и дискретизируется в моменты времени $t = k/f_s$,
что даёт последовательность принятых отсчётов $\left\{ r_k \right\}_{k \in \mathbb{Z}}$

Цель DSP чипа -- восстановить вектор $\mathbf{s}$ из полученного вектора $\mathbf{r}$, для чего используются следующие три блока:

\begin{itemize}
	\item Learned DBP (последовательность: линейный блок (FIR фильтр) + нелинейный блок)
	\item RRC-фильтр
	\item Коррекция фазового сдвига
\end{itemize}

Всю цепочку блоков DSP чипа можно записать в следующем виде:

\begin{equation}
	\mathbf{\hat{s}} = \exp{(-i\hat{\varphi})} \mathbf{M} \mathbf{f}_{\theta} (\mathbf{r}),
\end{equation}

где $\mathbf{M} \in \mathbb{R}^{N_{sym} \times n}$ -- циркулянтная матрица, представляющая операцию согласованной фильтрации (RRC-фильтр)

\section{Содержательная постановка задачи}

Основной проблемой DBP является большое вычислительное бремя, связанное с его реализацией в DSP чипе в реальном времени.
Поэтому, в наиболее общем виде, задача заключается в том, чтобы аппроксимировать численное решение NLSE уравнения в частных производных, используя как можно меньше вычислительных ресурсов.

Мы подходим к этой проблеме с точки зрения машинного обучения, применяя для оптимизации всех параметров в $\theta$ обучение с учителем, как описано в последнем разделе аналитического обзора литературы.

Обучение проводится с использованием оптимизатора Adam -- реализации стохастического градиентного спуска (SGD).

Выбранная функция потерь -- среднеквадратическая ошибка (MSE) между исходными и восстановленными символами, то есть $l(\mathbf{s}, \mathbf{\hat{s}}) = |\mathbf{s} - \mathbf{\hat{s}}|^2/N_{sym}$, где $|\mathbf{s}|^2 = \sum_{i=1}^{N_{sym}} |s_i|^2$, так как минимизация MSE эквивалентна увеличению SNR:

\begin{equation}
	\text{SNR} = N_{sym} \mathbb{E} \left( \left( | \mathbf{S} - \mathbf{\hat{S}} | \right)^{-1} \right)
\end{equation}

Важно отметить следующие два момента, которые будут использоваться во время дальнейшей работы:

\begin{enumerate}
	\item Для вычисления размеров шагов $\delta_i$ (при заданном числе шагов на участок, StPS, steps per span) будет использоваться логарифмическая эвристика из статьи~\cite{Zhang2013};
	\item Начальные коэффициенты FIR фильтров будут считаться с помощью метода наименьших квадратов на основе статьи~\cite{Sheikh2016};
\end{enumerate}

Программирование и обучение модели производится посредством библиотек языка программирования Python tensorflow и numpy.

\section{Математическая постановка задачи}

Канал: $\mathcal{H}: \mathbb{C}^n \to \mathbb{C}^n$ -- оператор распространения импульса в оптоволокне, описываемый нелинейным уравнением Шрёдингера (NLSE)

Переданный сигнал: $\mathbf{x} \in \mathbb{C}^n$

Принятый сигнал: $\mathbf{y} = \mathcal{H}(\mathbf{x}) + \mathbf{n} \in \mathbb{C}^n$, где шум $\mathbf{n} \sim C\mathcal{N}(0, \sigma^2 \mathbf{I})$

Обучающая выборка: $\mathcal{D} = \left\{ (\mathbf{x}_i, \mathbf{y}_i) \right\}_{i=1}^{N}$

Модель Learned DBP: $\mathbf{\hat{x}} = \mathbf{f}_{\theta}(\mathbf{y})$, где $\theta = \left\{ \mathbf{h}^{(1)}, \ldots, \mathbf{h}^{(M)} \right\}$ -- коэффициенты $M$ симметричных FIR (finite impulse response) фильтров.

Функция потерь: $\mathcal{L}(\theta) = (1/N) \sum_{i=1}^{N} \ell (\mathbf{x}^{(i)}, \mathbf{f}_{\theta}(\mathbf{y}^{(i)}))$, где $ \ell (\mathbf{x}, \mathbf{\hat{x}}) = |\mathbf{x} - \mathbf{\hat{x}}|^2$

Тогда имеем задачу оптимизации в следующем виде

\begin{equation}
	\theta^* = \arg\min_{\theta} \,\,\, \mathcal{L} (\theta)
\end{equation}

\section{Описание датасета}

Датасет предоставлен директором исследовательской лабораторией оптоволоконных алгоритмов (Moscow Optic Algorithm Laboratory) Huawei, Павлом Плотниковым,
\href{https://ru.linkedin.com/in/pavel-plotnikov-8bb6202b}{ссылка на LinkedIn}.
Параметры оптического канала и описание его моделирования приведены в Главе 4.1.

Всего в датасете два столбца, входной и выходной сигналы.
Каждый оптический сигнал это комплексное число $A(t) = I(t) + iQ(t)$ -- комплексная огибающая поля, связанная с электрическим полем соотношением (1 поляризация)

\begin{equation}
	E(t) = \text{Re}\left[ A(t) \cdot \text{e}^{i\omega_0t} \right]
\end{equation}

Мощность оптического сигнала равна $P(t) = | A(t) |^2 = I^2 + Q^2$.

Фаза оптического сигнала равна $\varphi(t) = \text{arg}(A(t)) = \arctan{(Q / I)}$.
