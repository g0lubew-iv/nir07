\newpage
\begin{center}
	\textbf{\large 2. СПЕЦИАЛЬНАЯ ЧАСТЬ}
\end{center}
\refstepcounter{chapter}
\addcontentsline{toc}{chapter}{2. СПЕЦИАЛЬНАЯ ЧАСТЬ}

\section{Параметры моделируемого канала}

Рассмотрим комплекснозначный сигнал $x(t)$, который распространяется по оптическому волокну длины $L$ согласно Рисунку $4.1$.

\begin{figure}[!h]
	\begin{center}
		\includegraphics[width=300pt]{propogation.png}
		\caption{Распространение сигнала в одномодовом оптоволокне.}
		\label{nmp}
	\end{center}
\end{figure}

Нелинейное уравнение Шрёдингера описывает взаимосвязь между отправленным сигналом $x(t)=u(z=0,t)$ и полученным сигналом $y(t)=u(z=L,t)$:

\begin{equation}
	\frac{\partial u(z,t)}{\partial z} = - \frac{\alpha}{2} u(z,t) - \frac{i\beta_2}{2} \frac{\partial^2 u(z,t)}{\partial t^2} + i \gamma |u(z,t)|^2 u(z,t),
\end{equation}

где $\alpha$ -- коэффициент затухания сигнала, $\beta_2$ -- коэффициент хроматической дисперсии, $\gamma$ -- нелинейный коэффициент (ответственный за эффект Керра).

Моделирование распространения сигнала по long-haul волокну выполнено лабораторией цифровой обработки сигналов Huawei.
Моделирование выполнялось следующим образом: сигнал распространялся по закольцованному волокну длиной около $10$ километров большое количество раз ($>60$ в нашем случае), также вносились дополнительные шумы, имитирующие
Параметры моделирования выбирались следующими:

\begin{itemize}
	\item Высокая излучаемая мощность лазера: $>10$ дБм, длина волны $\lambda = 1550$ нм.
	\item Длина канала $L = 640$ км.
	\item Модуляция сигнала: квадратурная модуляция $16\text{QAM}$ ($1$ символ -- $4$ бита информации) и кодирование Грея (Gray mapping); диаграмма созвездия показана на Рисунке $4.2$.
	\item 1 поляризация сигнала (single-polarization)
	\item Коэффициент затухания в оптоволокне $\alpha = 0.2$ дБ/км.
	\item Линейный коэффициент хроматической дисперсии $\beta_2 = -21.683$ $\text{пс}^2$/км.
	\item Нелинейный коэффициент Керра $\gamma = 1.3$ рад / (Ват$\cdot$км)
\end{itemize}

\begin{figure}[!h]
	\begin{center}
		\includegraphics[width=300pt]{qam-gray.png}
		\caption{16QAM модуляция с Gray mapping~\cite{QAMarticle}.}
		\label{nmp}
	\end{center}
\end{figure}

\section{Содержательная постановка задачи}

Целью данной научно-исследовательской работы является исследование алгоритма обученного обратного распространения (LDBP)
для цифровой компенсации нелинейных искажений в волоконно-оптических системах связи.

Сформированная теоретическая, методологическая и математическая база, достаточна для перехода к этапу программной реализации данного алгоритма.

На вход программе подаются вышеописанные параметры (одномодовой) оптической линии связи и датасет.
Посредством библиотек языка программирования Python tensorflow и numpy начинается процесс ???

Итоговая модель принимает на вход комплексное число $y(t)$ -- сигнал на приёмнике -- и возвращает число $x(t)$ -- восстановленный исходный сигнал.

\section{Параметры нейронной сети}

\section{Математическая постановка задачи}
