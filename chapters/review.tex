\newpage
\begin{center}
	\textbf{\large 3. АНАЛИТИЧЕСКИЙ ОБЗОР}
\end{center}
\refstepcounter{chapter}
\addcontentsline{toc}{chapter}{3. АНАЛИТИЧЕСКИЙ ОБЗОР}


\section{Влияние дальнодействия потенциала на критическое поведение}

Уравнения Максвелла запишем в следующем виде~\cite{10.1103/PhysRevLett.29.917}:

\begin{equation}
	\mathcal{H} / k_{\mathrm{B}} T=-K \sum_{\langle i j\rangle} \frac{s_{i} s_{j}}{r_{i j}^{d+\sigma}}
	\label{eq1}
\end{equation}

\section{Влияние дальнодействия потенциала на фазовые диаграммы и плавление}

В настоящий момент установлено, что 2D-сценарии плавления зависят от мягкости отталкивания, обеспечивая микроскопические сценарии 2D-плавления, описываемые в работах~\cite{10.3367/ufne.2017.06.038161, 10.3367/ufne.2018.04.038417}, что доказывает теория Березинского-Костерлица-Таулесса-Гальперина-Нельсона-Янга (БКТГНЯ), согласно которой плавление происходит через два непрерывных перехода с промежуточной гексатической фазой с квазидальним ориентационным порядком и ближним трансляционным порядком~\cite{10.1088/0022-3719/6/7/010, 10.1103/physrevlett.41.121, 10.1103/physrevb.19.2457, 10.1103/physrevb.19.1855}, плавление через фазовый переход первого рода, двухстадийное плавление, включающее непрерывный (Березинский-Костерлиц-Таулесс, БКТ) кристаллогексатический фазовый переход и фазовый переход первого рода между гексатической фазой и изотропной жидкостью.
Второй и третий сценарии присущи системам с короткодействующим (жестким) отталкиванием, тогда как первый наблюдался при мягком отталкивании между частицами.
Установлено, что мягкость отталкивания влияет на сценарии плавления, термодинамику и спектры возбуждения в монослойных системах.
Однако известно, что роль притяжения в сценарии плавления монослойных систем остается систематически неизученной.

\section{Цели и задачи магистерской работы}

\textbf{Цель работы} -- установить связь дальнодействия притяжения потенциала взаимодействия и спектров возбуждений с транспортными свойствами жидкостей, а также влияние на скорость нуклеации.

\textbf{Задачи работы:}
\begin{enumerate}
	\item Расчет фазовых диаграмм для 2D и 3D систем частиц, взаимодействующих посредством обобщенного потенциала Леннарда-Джонса с различными степенями притяжения.
	\item Адаптация метода кластеризации данных DBSCAN для изучения молекулярных систем и его сравнение с другими методами.
	\item Расчет и анализ транспортных свойств и коллективных возбуждений на жидкостных бинодалях.
	\item Применение нового метода распознавания фаз для изучения скорости нуклеации в переохлажденных системах Леннарда-Джонса с различным дальнодействием притяжения.
\end{enumerate}
