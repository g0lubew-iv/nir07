\newpage
\begin{center}
	\textbf{\large 2. ВВЕДЕНИЕ}
\end{center}
\refstepcounter{chapter}
\addcontentsline{toc}{chapter}{2. ВВЕДЕНИЕ}

Распространение светового сигнала в оптоволокне зависит от волоконных эффектов, искажающих передающийся сигнал. Эти эффекты хорошо моделируются NLSE, обыкновенно включающим в себя линейные эффекты затухания и GVD и нелинейный эффект Керра.

Для небольшой интенсивности лазера (до 80 нм???) нелинейный эффект Керра очень мал, а первые два эффекта хорошо компенсируются линейными фильтрами (например, ???). Однако на больших расстояниях (несколько сотен километров???) и, соответственно, больших мощностях лазера, нелинейные эффекты становятся очень явными.

К сожалению, общее аналитического решения NLSE не найдено. В связи с этим техники компенсации нелинейных эффектов очень ресурсоёмки.
