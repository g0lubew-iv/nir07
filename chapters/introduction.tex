\newpage
\begin{center}
	\textbf{\large 2. ВВЕДЕНИЕ}
\end{center}
\refstepcounter{chapter}
\addcontentsline{toc}{chapter}{2. ВВЕДЕНИЕ}

Оптоволоконные системы связи являются фундаментом глобальной информационной инфраструктуры XXI века, обеспечивая передачу подавляющего объема данных в интернете, мобильных сетях и межконтинентальных магистралях. Их ключевое преимущество — исключительная пропускная способность и низкие потери сигнала на больших расстояниях. Однако по мере роста спроса на трафик и увеличения скоростей передачи (сотни Гбит/с и выше на канал) физические ограничения оптического волокна становятся критическим барьером. В лонг-холл (длинных) кабелях, особенно подводных и трансконтинентальных, длительное распространение сигнала приводит к совокупному воздействию линейных (хроматическая дисперсия, затухание) и, что особенно важно, нелинейных эффектов Керра. Эти нелинейные искажения, возникающие из-за взаимодействия света с материалом волокна, не являются аддитивным шумом и их влияние принципиально зависит от самого передаваемого сигнала. Традиционные методы компенсации, такие как цифровая обработка сигналов (ЦОС) с линейным уравнением Шрёдингера, становятся неэффективными, делая нелинейность доминирующим фактором, ограничивающим дальность и емкость современных систем.

Существующие методы нелинейной компенсации, например, цифровая обратная распространение (DBP — Digital Backpropagation), сталкиваются с фундаментальными проблемами при практической реализации. Классический DBP требует точного знания параметров всей трассы, чрезвычайно высокой вычислительной сложности (особенно при учете поляризационной модовой дисперсии и эффектов Керра высших порядков) и становится практически неприменимым в системах с динамическим управлением спектром (flex-grid) или с реконфигурируемыми оптическими сетями (ROADM). Таким образом, возникает острая необходимость в разработке интеллектуальных, адаптивных и вычислительно эффективных алгоритмов, способных подавлять нелинейные помехи в реальном времени в условиях неполной информации о канале.
