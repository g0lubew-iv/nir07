\newpage
\begin{center}
	\textbf{ВВЕДЕНИЕ}
\end{center}
\refstepcounter{chapter}
\addcontentsline{toc}{chapter}{ВВЕДЕНИЕ}

Современные волоконно-оптические системы связи являются основой глобальной информационной инфраструктуры, обеспечивая передачу данных на эксабитных скоростях для трансконтинентальных расстояний~\cite{hecht2004city}.
Однако их дальнейшее развитие сталкивается с фундаментальным физическим ограничением -- нелинейными эффектами, возникающими при распространении мощного оптического импульса в оптическом волокне.
Данные эффекты, такие как фазовая самомодуляция, кросс-модуляция и четырехволновое смешение, приводят к нелинейным искажениям сигнала, которые традиционные методы компенсации, основанные на линеаризованных моделях, не способны полностью устранить.

Одним из наиболее точных методов компенсации нелинейных искажений является цифровой алгоритм обратного распространения -- численное решение нелинейного уравнения Шрёдингера с инвертированными коэффициентами.
Классический DBP требует решения NLSE с использованием методов расщепления Фурье, что связано с очень большой вычислительной сложностью.
Это делает его практическое применение в реальных системах с высокой скоростью передачи данных экономически и технически затруднительным,
поэтому в современных чипах цифровой компенсации искажений применяют оптимизированные версии DBP, зачастую совмещённые с достижениями теории возмущений.

В силу этих причин исследовательские телекомуникационные лаборатории заинтересованы в разработке различных методов аппроксимации DBP, в том числе с помощью технологий машинного обучения.

Данная работа исследует практическую реализацию алгоритма Learned DBP, основанного на аналогии между архитектурами параметризованного SSFM и NN.
В частности, линейные шаги DBP заменяются на симметричные FIR фильтры, коэффициенты которых оптимизируются с использованием SGD.
