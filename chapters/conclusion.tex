\newpage
\begin{center}
	\textbf{ЗАКЛЮЧЕНИЕ}
\end{center}
\refstepcounter{chapter}
\addcontentsline{toc}{chapter}{ЗАКЛЮЧЕНИЕ}

В рамках КНИР проведен анализ предметной и проблемной областей, изучена проблематика и различные подходы к алгоритму цифрового обратного распространения.

Установлена принципиальная возможность применения методов машинного обучения для решения физических задач в оптической связи.
Алгоритм Learned DBP представляет собой параметризацию хорошо изученного численного метода (Split-Step Fourier Method, SSFM), что обеспечивает интерпретируемость и возможность физически обоснованной инициализации.

Выявлено ключевое преимущество LDBP перед классическим DBP: способность находить компактные и эффективные аппроксимации обратного оператора распространения за счёт: замены точных операторов дисперсии на короткие обучаемые FIR-фильтры и использования техник сжатия моделей (pruning) для пошагового сокращения длины фильтров в процессе обучения.

Сформулирована математическая постановка задачи как задачи оптимизации MSE функции потерь.

По результатам анализа экспериментальных данных Huawei проведена декомпозиция и физическая интерпретация предоставленного датасета, определены ключевые параметры сигнала для последующего моделирования и обучения.

В ходе выполнения работы сформированы компетенции по анализу литературных источников, постановке математической и содержательной задач, пониманию устройства методов машинного обучения, а также получены знания в области нелинейной оптики.

В дальнейшем, в рамках выпускной квалификационной работы, планируется реализовать выбранный Lerned DBP метод на основе симметричных FIR фильтров и параметризованного SSFM
и сравнить результаты его работы с такими классическими методами, как PBM и DBP.

Таким образом, на текущем этапе цель КНИР можно считать частично достигнутой, поскольку продолжается работа над написанием кода.
